
% Default to the notebook output style

    


% Inherit from the specified cell style.




    
\documentclass[11pt]{article}

    
    
    \usepackage[T1]{fontenc}
    % Nicer default font (+ math font) than Computer Modern for most use cases
    \usepackage{mathpazo}

    % Basic figure setup, for now with no caption control since it's done
    % automatically by Pandoc (which extracts ![](path) syntax from Markdown).
    \usepackage{graphicx}
    % We will generate all images so they have a width \maxwidth. This means
    % that they will get their normal width if they fit onto the page, but
    % are scaled down if they would overflow the margins.
    \makeatletter
    \def\maxwidth{\ifdim\Gin@nat@width>\linewidth\linewidth
    \else\Gin@nat@width\fi}
    \makeatother
    \let\Oldincludegraphics\includegraphics
    % Set max figure width to be 80% of text width, for now hardcoded.
    \renewcommand{\includegraphics}[1]{\Oldincludegraphics[width=.8\maxwidth]{#1}}
    % Ensure that by default, figures have no caption (until we provide a
    % proper Figure object with a Caption API and a way to capture that
    % in the conversion process - todo).
    \usepackage{caption}
    \DeclareCaptionLabelFormat{nolabel}{}
    \captionsetup{labelformat=nolabel}

    \usepackage{adjustbox} % Used to constrain images to a maximum size 
    \usepackage{xcolor} % Allow colors to be defined
    \usepackage{enumerate} % Needed for markdown enumerations to work
    \usepackage{geometry} % Used to adjust the document margins
    \usepackage{amsmath} % Equations
    \usepackage{amssymb} % Equations
    \usepackage{textcomp} % defines textquotesingle
    % Hack from http://tex.stackexchange.com/a/47451/13684:
    \AtBeginDocument{%
        \def\PYZsq{\textquotesingle}% Upright quotes in Pygmentized code
    }
    \usepackage{upquote} % Upright quotes for verbatim code
    \usepackage{eurosym} % defines \euro
    \usepackage[mathletters]{ucs} % Extended unicode (utf-8) support
    \usepackage[utf8x]{inputenc} % Allow utf-8 characters in the tex document
    \usepackage{fancyvrb} % verbatim replacement that allows latex
    \usepackage{grffile} % extends the file name processing of package graphics 
                         % to support a larger range 
    % The hyperref package gives us a pdf with properly built
    % internal navigation ('pdf bookmarks' for the table of contents,
    % internal cross-reference links, web links for URLs, etc.)
    \usepackage{hyperref}
    \usepackage{longtable} % longtable support required by pandoc >1.10
    \usepackage{booktabs}  % table support for pandoc > 1.12.2
    \usepackage[inline]{enumitem} % IRkernel/repr support (it uses the enumerate* environment)
    \usepackage[normalem]{ulem} % ulem is needed to support strikethroughs (\sout)
                                % normalem makes italics be italics, not underlines
    

    
    
    % Colors for the hyperref package
    \definecolor{urlcolor}{rgb}{0,.145,.698}
    \definecolor{linkcolor}{rgb}{.71,0.21,0.01}
    \definecolor{citecolor}{rgb}{.12,.54,.11}

    % ANSI colors
    \definecolor{ansi-black}{HTML}{3E424D}
    \definecolor{ansi-black-intense}{HTML}{282C36}
    \definecolor{ansi-red}{HTML}{E75C58}
    \definecolor{ansi-red-intense}{HTML}{B22B31}
    \definecolor{ansi-green}{HTML}{00A250}
    \definecolor{ansi-green-intense}{HTML}{007427}
    \definecolor{ansi-yellow}{HTML}{DDB62B}
    \definecolor{ansi-yellow-intense}{HTML}{B27D12}
    \definecolor{ansi-blue}{HTML}{208FFB}
    \definecolor{ansi-blue-intense}{HTML}{0065CA}
    \definecolor{ansi-magenta}{HTML}{D160C4}
    \definecolor{ansi-magenta-intense}{HTML}{A03196}
    \definecolor{ansi-cyan}{HTML}{60C6C8}
    \definecolor{ansi-cyan-intense}{HTML}{258F8F}
    \definecolor{ansi-white}{HTML}{C5C1B4}
    \definecolor{ansi-white-intense}{HTML}{A1A6B2}

    % commands and environments needed by pandoc snippets
    % extracted from the output of `pandoc -s`
    \providecommand{\tightlist}{%
      \setlength{\itemsep}{0pt}\setlength{\parskip}{0pt}}
    \DefineVerbatimEnvironment{Highlighting}{Verbatim}{commandchars=\\\{\}}
    % Add ',fontsize=\small' for more characters per line
    \newenvironment{Shaded}{}{}
    \newcommand{\KeywordTok}[1]{\textcolor[rgb]{0.00,0.44,0.13}{\textbf{{#1}}}}
    \newcommand{\DataTypeTok}[1]{\textcolor[rgb]{0.56,0.13,0.00}{{#1}}}
    \newcommand{\DecValTok}[1]{\textcolor[rgb]{0.25,0.63,0.44}{{#1}}}
    \newcommand{\BaseNTok}[1]{\textcolor[rgb]{0.25,0.63,0.44}{{#1}}}
    \newcommand{\FloatTok}[1]{\textcolor[rgb]{0.25,0.63,0.44}{{#1}}}
    \newcommand{\CharTok}[1]{\textcolor[rgb]{0.25,0.44,0.63}{{#1}}}
    \newcommand{\StringTok}[1]{\textcolor[rgb]{0.25,0.44,0.63}{{#1}}}
    \newcommand{\CommentTok}[1]{\textcolor[rgb]{0.38,0.63,0.69}{\textit{{#1}}}}
    \newcommand{\OtherTok}[1]{\textcolor[rgb]{0.00,0.44,0.13}{{#1}}}
    \newcommand{\AlertTok}[1]{\textcolor[rgb]{1.00,0.00,0.00}{\textbf{{#1}}}}
    \newcommand{\FunctionTok}[1]{\textcolor[rgb]{0.02,0.16,0.49}{{#1}}}
    \newcommand{\RegionMarkerTok}[1]{{#1}}
    \newcommand{\ErrorTok}[1]{\textcolor[rgb]{1.00,0.00,0.00}{\textbf{{#1}}}}
    \newcommand{\NormalTok}[1]{{#1}}
    
    % Additional commands for more recent versions of Pandoc
    \newcommand{\ConstantTok}[1]{\textcolor[rgb]{0.53,0.00,0.00}{{#1}}}
    \newcommand{\SpecialCharTok}[1]{\textcolor[rgb]{0.25,0.44,0.63}{{#1}}}
    \newcommand{\VerbatimStringTok}[1]{\textcolor[rgb]{0.25,0.44,0.63}{{#1}}}
    \newcommand{\SpecialStringTok}[1]{\textcolor[rgb]{0.73,0.40,0.53}{{#1}}}
    \newcommand{\ImportTok}[1]{{#1}}
    \newcommand{\DocumentationTok}[1]{\textcolor[rgb]{0.73,0.13,0.13}{\textit{{#1}}}}
    \newcommand{\AnnotationTok}[1]{\textcolor[rgb]{0.38,0.63,0.69}{\textbf{\textit{{#1}}}}}
    \newcommand{\CommentVarTok}[1]{\textcolor[rgb]{0.38,0.63,0.69}{\textbf{\textit{{#1}}}}}
    \newcommand{\VariableTok}[1]{\textcolor[rgb]{0.10,0.09,0.49}{{#1}}}
    \newcommand{\ControlFlowTok}[1]{\textcolor[rgb]{0.00,0.44,0.13}{\textbf{{#1}}}}
    \newcommand{\OperatorTok}[1]{\textcolor[rgb]{0.40,0.40,0.40}{{#1}}}
    \newcommand{\BuiltInTok}[1]{{#1}}
    \newcommand{\ExtensionTok}[1]{{#1}}
    \newcommand{\PreprocessorTok}[1]{\textcolor[rgb]{0.74,0.48,0.00}{{#1}}}
    \newcommand{\AttributeTok}[1]{\textcolor[rgb]{0.49,0.56,0.16}{{#1}}}
    \newcommand{\InformationTok}[1]{\textcolor[rgb]{0.38,0.63,0.69}{\textbf{\textit{{#1}}}}}
    \newcommand{\WarningTok}[1]{\textcolor[rgb]{0.38,0.63,0.69}{\textbf{\textit{{#1}}}}}
    
    
    % Define a nice break command that doesn't care if a line doesn't already
    % exist.
    \def\br{\hspace*{\fill} \\* }
    % Math Jax compatability definitions
    \def\gt{>}
    \def\lt{<}
    % Document parameters
    \title{HW2-LinearRegression-Worked}
    
    
    

    % Pygments definitions
    
\makeatletter
\def\PY@reset{\let\PY@it=\relax \let\PY@bf=\relax%
    \let\PY@ul=\relax \let\PY@tc=\relax%
    \let\PY@bc=\relax \let\PY@ff=\relax}
\def\PY@tok#1{\csname PY@tok@#1\endcsname}
\def\PY@toks#1+{\ifx\relax#1\empty\else%
    \PY@tok{#1}\expandafter\PY@toks\fi}
\def\PY@do#1{\PY@bc{\PY@tc{\PY@ul{%
    \PY@it{\PY@bf{\PY@ff{#1}}}}}}}
\def\PY#1#2{\PY@reset\PY@toks#1+\relax+\PY@do{#2}}

\expandafter\def\csname PY@tok@ow\endcsname{\let\PY@bf=\textbf\def\PY@tc##1{\textcolor[rgb]{0.67,0.13,1.00}{##1}}}
\expandafter\def\csname PY@tok@sc\endcsname{\def\PY@tc##1{\textcolor[rgb]{0.73,0.13,0.13}{##1}}}
\expandafter\def\csname PY@tok@sb\endcsname{\def\PY@tc##1{\textcolor[rgb]{0.73,0.13,0.13}{##1}}}
\expandafter\def\csname PY@tok@nd\endcsname{\def\PY@tc##1{\textcolor[rgb]{0.67,0.13,1.00}{##1}}}
\expandafter\def\csname PY@tok@go\endcsname{\def\PY@tc##1{\textcolor[rgb]{0.53,0.53,0.53}{##1}}}
\expandafter\def\csname PY@tok@gd\endcsname{\def\PY@tc##1{\textcolor[rgb]{0.63,0.00,0.00}{##1}}}
\expandafter\def\csname PY@tok@nb\endcsname{\def\PY@tc##1{\textcolor[rgb]{0.00,0.50,0.00}{##1}}}
\expandafter\def\csname PY@tok@nt\endcsname{\let\PY@bf=\textbf\def\PY@tc##1{\textcolor[rgb]{0.00,0.50,0.00}{##1}}}
\expandafter\def\csname PY@tok@cm\endcsname{\let\PY@it=\textit\def\PY@tc##1{\textcolor[rgb]{0.25,0.50,0.50}{##1}}}
\expandafter\def\csname PY@tok@kr\endcsname{\let\PY@bf=\textbf\def\PY@tc##1{\textcolor[rgb]{0.00,0.50,0.00}{##1}}}
\expandafter\def\csname PY@tok@gr\endcsname{\def\PY@tc##1{\textcolor[rgb]{1.00,0.00,0.00}{##1}}}
\expandafter\def\csname PY@tok@s\endcsname{\def\PY@tc##1{\textcolor[rgb]{0.73,0.13,0.13}{##1}}}
\expandafter\def\csname PY@tok@c1\endcsname{\let\PY@it=\textit\def\PY@tc##1{\textcolor[rgb]{0.25,0.50,0.50}{##1}}}
\expandafter\def\csname PY@tok@gi\endcsname{\def\PY@tc##1{\textcolor[rgb]{0.00,0.63,0.00}{##1}}}
\expandafter\def\csname PY@tok@sh\endcsname{\def\PY@tc##1{\textcolor[rgb]{0.73,0.13,0.13}{##1}}}
\expandafter\def\csname PY@tok@c\endcsname{\let\PY@it=\textit\def\PY@tc##1{\textcolor[rgb]{0.25,0.50,0.50}{##1}}}
\expandafter\def\csname PY@tok@mh\endcsname{\def\PY@tc##1{\textcolor[rgb]{0.40,0.40,0.40}{##1}}}
\expandafter\def\csname PY@tok@gp\endcsname{\let\PY@bf=\textbf\def\PY@tc##1{\textcolor[rgb]{0.00,0.00,0.50}{##1}}}
\expandafter\def\csname PY@tok@ge\endcsname{\let\PY@it=\textit}
\expandafter\def\csname PY@tok@kt\endcsname{\def\PY@tc##1{\textcolor[rgb]{0.69,0.00,0.25}{##1}}}
\expandafter\def\csname PY@tok@kd\endcsname{\let\PY@bf=\textbf\def\PY@tc##1{\textcolor[rgb]{0.00,0.50,0.00}{##1}}}
\expandafter\def\csname PY@tok@gu\endcsname{\let\PY@bf=\textbf\def\PY@tc##1{\textcolor[rgb]{0.50,0.00,0.50}{##1}}}
\expandafter\def\csname PY@tok@mo\endcsname{\def\PY@tc##1{\textcolor[rgb]{0.40,0.40,0.40}{##1}}}
\expandafter\def\csname PY@tok@s2\endcsname{\def\PY@tc##1{\textcolor[rgb]{0.73,0.13,0.13}{##1}}}
\expandafter\def\csname PY@tok@no\endcsname{\def\PY@tc##1{\textcolor[rgb]{0.53,0.00,0.00}{##1}}}
\expandafter\def\csname PY@tok@ch\endcsname{\let\PY@it=\textit\def\PY@tc##1{\textcolor[rgb]{0.25,0.50,0.50}{##1}}}
\expandafter\def\csname PY@tok@fm\endcsname{\def\PY@tc##1{\textcolor[rgb]{0.00,0.00,1.00}{##1}}}
\expandafter\def\csname PY@tok@w\endcsname{\def\PY@tc##1{\textcolor[rgb]{0.73,0.73,0.73}{##1}}}
\expandafter\def\csname PY@tok@kc\endcsname{\let\PY@bf=\textbf\def\PY@tc##1{\textcolor[rgb]{0.00,0.50,0.00}{##1}}}
\expandafter\def\csname PY@tok@o\endcsname{\def\PY@tc##1{\textcolor[rgb]{0.40,0.40,0.40}{##1}}}
\expandafter\def\csname PY@tok@gh\endcsname{\let\PY@bf=\textbf\def\PY@tc##1{\textcolor[rgb]{0.00,0.00,0.50}{##1}}}
\expandafter\def\csname PY@tok@cs\endcsname{\let\PY@it=\textit\def\PY@tc##1{\textcolor[rgb]{0.25,0.50,0.50}{##1}}}
\expandafter\def\csname PY@tok@ni\endcsname{\let\PY@bf=\textbf\def\PY@tc##1{\textcolor[rgb]{0.60,0.60,0.60}{##1}}}
\expandafter\def\csname PY@tok@si\endcsname{\let\PY@bf=\textbf\def\PY@tc##1{\textcolor[rgb]{0.73,0.40,0.53}{##1}}}
\expandafter\def\csname PY@tok@na\endcsname{\def\PY@tc##1{\textcolor[rgb]{0.49,0.56,0.16}{##1}}}
\expandafter\def\csname PY@tok@sa\endcsname{\def\PY@tc##1{\textcolor[rgb]{0.73,0.13,0.13}{##1}}}
\expandafter\def\csname PY@tok@sd\endcsname{\let\PY@it=\textit\def\PY@tc##1{\textcolor[rgb]{0.73,0.13,0.13}{##1}}}
\expandafter\def\csname PY@tok@vm\endcsname{\def\PY@tc##1{\textcolor[rgb]{0.10,0.09,0.49}{##1}}}
\expandafter\def\csname PY@tok@nc\endcsname{\let\PY@bf=\textbf\def\PY@tc##1{\textcolor[rgb]{0.00,0.00,1.00}{##1}}}
\expandafter\def\csname PY@tok@gt\endcsname{\def\PY@tc##1{\textcolor[rgb]{0.00,0.27,0.87}{##1}}}
\expandafter\def\csname PY@tok@nv\endcsname{\def\PY@tc##1{\textcolor[rgb]{0.10,0.09,0.49}{##1}}}
\expandafter\def\csname PY@tok@s1\endcsname{\def\PY@tc##1{\textcolor[rgb]{0.73,0.13,0.13}{##1}}}
\expandafter\def\csname PY@tok@se\endcsname{\let\PY@bf=\textbf\def\PY@tc##1{\textcolor[rgb]{0.73,0.40,0.13}{##1}}}
\expandafter\def\csname PY@tok@ne\endcsname{\let\PY@bf=\textbf\def\PY@tc##1{\textcolor[rgb]{0.82,0.25,0.23}{##1}}}
\expandafter\def\csname PY@tok@cpf\endcsname{\let\PY@it=\textit\def\PY@tc##1{\textcolor[rgb]{0.25,0.50,0.50}{##1}}}
\expandafter\def\csname PY@tok@bp\endcsname{\def\PY@tc##1{\textcolor[rgb]{0.00,0.50,0.00}{##1}}}
\expandafter\def\csname PY@tok@mf\endcsname{\def\PY@tc##1{\textcolor[rgb]{0.40,0.40,0.40}{##1}}}
\expandafter\def\csname PY@tok@vg\endcsname{\def\PY@tc##1{\textcolor[rgb]{0.10,0.09,0.49}{##1}}}
\expandafter\def\csname PY@tok@sr\endcsname{\def\PY@tc##1{\textcolor[rgb]{0.73,0.40,0.53}{##1}}}
\expandafter\def\csname PY@tok@gs\endcsname{\let\PY@bf=\textbf}
\expandafter\def\csname PY@tok@k\endcsname{\let\PY@bf=\textbf\def\PY@tc##1{\textcolor[rgb]{0.00,0.50,0.00}{##1}}}
\expandafter\def\csname PY@tok@nf\endcsname{\def\PY@tc##1{\textcolor[rgb]{0.00,0.00,1.00}{##1}}}
\expandafter\def\csname PY@tok@kn\endcsname{\let\PY@bf=\textbf\def\PY@tc##1{\textcolor[rgb]{0.00,0.50,0.00}{##1}}}
\expandafter\def\csname PY@tok@dl\endcsname{\def\PY@tc##1{\textcolor[rgb]{0.73,0.13,0.13}{##1}}}
\expandafter\def\csname PY@tok@nl\endcsname{\def\PY@tc##1{\textcolor[rgb]{0.63,0.63,0.00}{##1}}}
\expandafter\def\csname PY@tok@il\endcsname{\def\PY@tc##1{\textcolor[rgb]{0.40,0.40,0.40}{##1}}}
\expandafter\def\csname PY@tok@mb\endcsname{\def\PY@tc##1{\textcolor[rgb]{0.40,0.40,0.40}{##1}}}
\expandafter\def\csname PY@tok@mi\endcsname{\def\PY@tc##1{\textcolor[rgb]{0.40,0.40,0.40}{##1}}}
\expandafter\def\csname PY@tok@cp\endcsname{\def\PY@tc##1{\textcolor[rgb]{0.74,0.48,0.00}{##1}}}
\expandafter\def\csname PY@tok@err\endcsname{\def\PY@bc##1{\setlength{\fboxsep}{0pt}\fcolorbox[rgb]{1.00,0.00,0.00}{1,1,1}{\strut ##1}}}
\expandafter\def\csname PY@tok@m\endcsname{\def\PY@tc##1{\textcolor[rgb]{0.40,0.40,0.40}{##1}}}
\expandafter\def\csname PY@tok@ss\endcsname{\def\PY@tc##1{\textcolor[rgb]{0.10,0.09,0.49}{##1}}}
\expandafter\def\csname PY@tok@sx\endcsname{\def\PY@tc##1{\textcolor[rgb]{0.00,0.50,0.00}{##1}}}
\expandafter\def\csname PY@tok@kp\endcsname{\def\PY@tc##1{\textcolor[rgb]{0.00,0.50,0.00}{##1}}}
\expandafter\def\csname PY@tok@vc\endcsname{\def\PY@tc##1{\textcolor[rgb]{0.10,0.09,0.49}{##1}}}
\expandafter\def\csname PY@tok@nn\endcsname{\let\PY@bf=\textbf\def\PY@tc##1{\textcolor[rgb]{0.00,0.00,1.00}{##1}}}
\expandafter\def\csname PY@tok@vi\endcsname{\def\PY@tc##1{\textcolor[rgb]{0.10,0.09,0.49}{##1}}}

\def\PYZbs{\char`\\}
\def\PYZus{\char`\_}
\def\PYZob{\char`\{}
\def\PYZcb{\char`\}}
\def\PYZca{\char`\^}
\def\PYZam{\char`\&}
\def\PYZlt{\char`\<}
\def\PYZgt{\char`\>}
\def\PYZsh{\char`\#}
\def\PYZpc{\char`\%}
\def\PYZdl{\char`\$}
\def\PYZhy{\char`\-}
\def\PYZsq{\char`\'}
\def\PYZdq{\char`\"}
\def\PYZti{\char`\~}
% for compatibility with earlier versions
\def\PYZat{@}
\def\PYZlb{[}
\def\PYZrb{]}
\makeatother


    % Exact colors from NB
    \definecolor{incolor}{rgb}{0.0, 0.0, 0.5}
    \definecolor{outcolor}{rgb}{0.545, 0.0, 0.0}



    
    % Prevent overflowing lines due to hard-to-break entities
    \sloppy 
    % Setup hyperref package
    \hypersetup{
      breaklinks=true,  % so long urls are correctly broken across lines
      colorlinks=true,
      urlcolor=urlcolor,
      linkcolor=linkcolor,
      citecolor=citecolor,
      }
    % Slightly bigger margins than the latex defaults
    
    \geometry{verbose,tmargin=1in,bmargin=1in,lmargin=1in,rmargin=1in}
    
    

    \begin{document}
    
    
    \maketitle
    
    

    
    \section{Homework 2 - Linear
Regression}\label{homework-2---linear-regression}

In this assignment, you will develop a simple linear regression, a
multiple linear regression, and a logistic regression.

You will need two datasets for this assignment: 1) Stream Temperatures -
http://explorer.bee.oregonstate.edu/Topic/Modeling/Data/StreamTemp.xlsx
2) Fish Presence/Absence -
http://explorer.bee.oregonstate.edu/Topic/Modeling/Data/FishPresenceAbsence.xlsx

    \subsection{Problem 1}\label{problem-1}

** Using the Stream Temperature datasets available on Canvas, develop a
linear regression model that predicts the Maximum Daily Stream
Temperature as a linear function of Maximum Daily Air Temperature. The
model will be of the form: **

MaxStreamTemp = b1 + b2*MaxAirTemp

    *** READ THIS BEFORE STARTING THIS ASSIGNMENT!!!***

*** The exercises below build on each other, and must be done in
sequence, first to last, since variables defined in earlier code blocks
are often used in later codeblocks. *** If your Python is abit rusty,
you might look at
\href{http://nbviewer.jupyter.org/gist/rpmuller/5920182}{this site} for
a brief "crash course" in Python for data analysis.

In the notebook, we are going to do some simple linear and multiple
regressions. Regressions need data, specifically data containing one or
more \textbf{\emph{response}} variables, and one or more independent
\textbf{\emph{explanatory}} variables. For this problem we will utilize
a Stream Temperature dataset, containing a time series of measurements
for a stream in Western Oregon, colelcted from a USGS gaging station.
The dataset include day, day of year, discharge at the stream location,
and a variety of additional climate variables

Your first task is to write a python program that 1) reads this dataset
into your program, 2) prints each column label to the console, and 3)
prints the number of rows in the dataset to the console.

We will utilize the \textbf{\emph{Pandas}} library to simplify your
task. If you aren''t familiar with \textbf{\emph{Pandas}}, or need a
refresher, you might look at this
\href{http://pandas.pydata.org/pandas-docs/stable/10min.html}{10-minute
tutorial on Pandas}.

Specifically, because the data is in an Excel file, you will need
pandas' \texttt{read\_excel()} function to read the dataset from a
remote web server (so you need to be online - alternatively, you can
download the file locally). The docs for \texttt{read\_excel()} are
\href{http://pandas.pydata.org/pandas-docs/stable/io.html}{available
here}. \texttt{read\_excel()} returns a \emph{DataFrame}, an object that
can be thought of as a database table with rows and columns.

The docs for DataFrames are
\href{http://pandas.pydata.org/pandas-docs/stable/generated/pandas.DataFrame.html}{available
here}

\subsubsection{Step 1. Read in the
Dataset}\label{step-1.-read-in-the-dataset}

Our first step in this exercise is to read some data. In the code block
below, add code to get the dataframe from the Excel file, utilizing
\textbf{\emph{Pandas}} \texttt{read\_excel()} function. Once your loaded
the dataset, print the dataframe to the console. Refer to the docs for
details on using \texttt{read\_excel()}.

    \begin{Verbatim}[commandchars=\\\{\}]
{\color{incolor}In [{\color{incolor}2}]:} \PY{k+kn}{import} \PY{n+nn}{pandas} \PY{k}{as} \PY{n+nn}{pd}
        
        \PY{c+c1}{\PYZsh{} Step 1. Read the excel file at http://explorer.bee.oregonstate.edu/Topic/Modeling/Data/StreamTemp.xlsx}
        \PY{c+c1}{\PYZsh{}         into a Pandas dataframe}
        \PY{c+c1}{\PYZsh{} start by loading a dataset from a CSV file}
        \PY{n}{df} \PY{o}{=} \PY{n}{pd}\PY{o}{.}\PY{n}{read\PYZus{}excel}\PY{p}{(} \PY{l+s+s1}{\PYZsq{}}\PY{l+s+s1}{http://explorer.bee.oregonstate.edu/Topic/Modeling/Data/StreamTemp.xlsx}\PY{l+s+s1}{\PYZsq{}} \PY{p}{)}
\end{Verbatim}


    \subsubsection{Step 2. Extract dataset
information}\label{step-2.-extract-dataset-information}

Once you have a dataframe populated with the data from the Excel
spreadsheet, you will want to \textbf{get a list of the columns in the
dataset}. Dataframes have a property 'columns' that returns information
about the columns in the dataset, including a list of column labels. You
can get this list using
\emph{\texttt{dataframe}}\texttt{.columns.values.tolist()}, where
\emph{dataframe} is the name of the dataframe returned from your
\texttt{read\_excel()} call. This will return a Python list with the
column labels. A short-hand version that does the same thing is:
\texttt{list(}\emph{\texttt{dataframe}}\texttt{)}.

You'll also want to pull the column data for the columns of interest
into individual arrays. This is easy, just use something like
\texttt{colarray\ =\ df{[}\textquotesingle{}MaxStreamTempC\textquotesingle{}{]}.value}
where \emph{df} is your dataframe containing the data, and
\emph{colArray} is the array receiving the data.

In the code block below, using your \emph{dataframe} from above:

\begin{itemize}
\item
  Extract the two columns of data of interest ("MaxStreamTempC",
  "MaxAirTempC") into two \emph{Series} objects containing the columns'
  data. See
  \href{http://www.datacarpentry.org/python-ecology-lesson/02-index-slice-subset/}{this
  example} if you need help.
\item
  Convert them to numpy arrays (that will simplify plotting). This is
  easy to do - once you have Series, use it's \texttt{.value} property
  to access a numpy array with the data.
\end{itemize}

Also, print the list of the column names in the \emph{dataframe} you
generated in the code block above - this will ensure you are reading the
file correctly.

    \begin{Verbatim}[commandchars=\\\{\}]
{\color{incolor}In [{\color{incolor}3}]:} \PY{c+c1}{\PYZsh{} Step 2. Extract the column labels from the dataframe and print them to the console}
        \PY{n+nb}{print}\PY{p}{(} \PY{n}{df}\PY{o}{.}\PY{n}{columns}\PY{o}{.}\PY{n}{values} \PY{p}{)}
        \PY{n}{xObs} \PY{o}{=} \PY{n}{df}\PY{p}{[} \PY{l+s+s1}{\PYZsq{}}\PY{l+s+s1}{MaxAirTempC}\PY{l+s+s1}{\PYZsq{}} \PY{p}{]}\PY{o}{.}\PY{n}{values}               \PY{c+c1}{\PYZsh{} extract x observations (independent variable)}
        \PY{n}{yObs} \PY{o}{=} \PY{n}{df}\PY{p}{[} \PY{l+s+s1}{\PYZsq{}}\PY{l+s+s1}{MaxStreamTempC}\PY{l+s+s1}{\PYZsq{}} \PY{p}{]}\PY{o}{.}\PY{n}{values}            \PY{c+c1}{\PYZsh{} extract y observations (dependent variable)}
\end{Verbatim}


    \begin{Verbatim}[commandchars=\\\{\}]
['StationID' 'Date' 'Latitude' 'Longitude' 'StreamDischargeCFS'
 'MinAirTempC' 'MaxAirTempC' 'MeanAirTempC' 'DayOfYear' 'MaxStreamTempC']

    \end{Verbatim}

    \subsubsection{Step 3. Run Regression}\label{step-3.-run-regression}

You may want to review the "Linear Regression in Python" PowerPoint
available on the course website - it steps the the step required to
conduct a regression on this data. You can use either the
\emph{\texttt{polyfit()}} function in numpy, or the
\emph{\texttt{linregress()}} function in the scipy.stats package.

    \begin{Verbatim}[commandchars=\\\{\}]
{\color{incolor}In [{\color{incolor}6}]:} \PY{c+c1}{\PYZsh{} Step 3 \PYZhy{} Run regression \PYZhy{} this is the linregress() version.  Could also use polyfit(), but linregress will calcuate the }
        \PY{c+c1}{\PYZsh{} stats automatically.}
        \PY{k+kn}{from} \PY{n+nn}{scipy} \PY{k}{import} \PY{n}{stats}
        \PY{k+kn}{import} \PY{n+nn}{numpy} \PY{k}{as} \PY{n+nn}{np}
        
        \PY{n}{slope}\PY{p}{,} \PY{n}{intercept}\PY{p}{,} \PY{n}{r\PYZus{}value}\PY{p}{,} \PY{n}{p\PYZus{}value}\PY{p}{,} \PY{n}{std\PYZus{}err} \PY{o}{=} \PY{n}{stats}\PY{o}{.}\PY{n}{linregress}\PY{p}{(}\PY{n}{xObs}\PY{p}{,}\PY{n}{yObs}\PY{p}{)}
        \PY{n+nb}{print}\PY{p}{(} \PY{l+s+s2}{\PYZdq{}}\PY{l+s+s2}{linregress(): Slope: }\PY{l+s+si}{\PYZob{}:.3\PYZcb{}}\PY{l+s+s2}{, Intercept: }\PY{l+s+si}{\PYZob{}:.3\PYZcb{}}\PY{l+s+s2}{, r2: }\PY{l+s+si}{\PYZob{}:.3\PYZcb{}}\PY{l+s+s2}{\PYZdq{}}\PY{o}{.}\PY{n}{format}\PY{p}{(} \PY{n}{slope}\PY{p}{,} \PY{n}{intercept}\PY{p}{,} \PY{n}{r\PYZus{}value}\PY{o}{*}\PY{o}{*}\PY{l+m+mi}{2} \PY{p}{)}\PY{p}{)}
        
        
        \PY{n}{coefs}\PY{o}{=}\PY{n}{np}\PY{o}{.}\PY{n}{polyfit}\PY{p}{(} \PY{n}{xObs}\PY{p}{,}\PY{n}{yObs}\PY{p}{,}\PY{l+m+mi}{1}\PY{p}{)}
        \PY{n+nb}{print}\PY{p}{(}\PY{n}{coefs}\PY{p}{)}
\end{Verbatim}


    \begin{Verbatim}[commandchars=\\\{\}]
linregress(): Slope: 0.567, Intercept: 3.09, r2: 0.717
[ 0.56725607  3.08726412]

    \end{Verbatim}

    \subsubsection{Step 4. Plot the Data}\label{step-4.-plot-the-data}

Now, we have a the observational data and a regression result. Let's
\textbf{generate a plot showing the data}. To do this, we will employ a
Python library called \textbf{\texttt{matplotlib}}, (actually, a
submodule called \textbf{\texttt{matplotlib.pylab}}), that provides a
powerful visualization library that is easy to use in your programs.

\begin{itemize}
\item
  For an overview of \textbf{\texttt{matplotlib}}, see
  \href{https://matplotlib.org/index.html}{the docs}, or go through
  \href{https://matplotlib.org/users/pyplot_tutorial.html}{the
  tutorial}.
  \href{https://github.com/jrjohansson/scientific-python-lectures/blob/master/Lecture-4-Matplotlib.ipynb}{This
  site} is a Jyputer notebook with an excellent intro to 2D and 3D
  plotting with \texttt{matplotlib}.
\item
  To see (a pretty amazing) gallery of \textbf{\texttt{matplotlib}}
  examples, complete with source code
  \href{https://matplotlib.org/gallery.html}{go here}.
\end{itemize}

In the code block below, using your \emph{dataframe} from above:

\begin{itemize}
\item
  Extract the two columns of data of interest ("MaxStreamTempC",
  "MaxAirTempC") into two \emph{Series} objects containing the columns'
  data. See
  \href{http://www.datacarpentry.org/python-ecology-lesson/02-index-slice-subset/}{this
  example} if you need help.
\item
  Convert them to numpy arrays (that will simplify plotting). This is
  easy to do - once you have Series, use it's \texttt{.value} property
  to access a numpy array with the data.
\item
  Add the data series to your \emph{plt} object below using the
  plt.plot() function (see the docs
  \href{https://matplotlib.org/api/pyplot_api.html?highlight=plot\#matplotlib.pyplot.plot}{here},
  or this
  \href{https://www.datacamp.com/community/tutorials/matplotlib-tutorial-python}{easy
  to understand tutorial}.
\end{itemize}

    \begin{Verbatim}[commandchars=\\\{\}]
{\color{incolor}In [{\color{incolor}4}]:} \PY{k+kn}{import} \PY{n+nn}{matplotlib}\PY{n+nn}{.}\PY{n+nn}{pyplot} \PY{k}{as} \PY{n+nn}{plt}   \PY{c+c1}{\PYZsh{} import the pyplot module of matplotlib}
        
        \PY{n}{yModeled} \PY{o}{=} \PY{n}{slope} \PY{o}{*} \PY{n}{xObs} \PY{o}{+} \PY{n}{intercept}  \PY{c+c1}{\PYZsh{} generate modeled results for each observation}
        
        \PY{n}{plt}\PY{o}{.}\PY{n}{plot}\PY{p}{(}\PY{n}{xObs}\PY{p}{,}\PY{n}{yObs}\PY{p}{,}\PY{l+s+s1}{\PYZsq{}}\PY{l+s+s1}{o}\PY{l+s+s1}{\PYZsq{}}\PY{p}{,} \PY{n}{color}\PY{o}{=}\PY{l+s+s1}{\PYZsq{}}\PY{l+s+s1}{blue}\PY{l+s+s1}{\PYZsq{}} \PY{p}{)}   \PY{c+c1}{\PYZsh{} add observation series}
        \PY{n}{plt}\PY{o}{.}\PY{n}{plot}\PY{p}{(} \PY{n}{xObs}\PY{p}{,} \PY{n}{yModeled}\PY{p}{,} \PY{l+s+s1}{\PYZsq{}}\PY{l+s+s1}{\PYZhy{}}\PY{l+s+s1}{\PYZsq{}}\PY{p}{,} \PY{n}{color}\PY{o}{=}\PY{l+s+s1}{\PYZsq{}}\PY{l+s+s1}{red}\PY{l+s+s1}{\PYZsq{}} \PY{p}{)}    \PY{c+c1}{\PYZsh{}\PYZsh{} add modeled series}
        
        \PY{n}{plt}\PY{o}{.}\PY{n}{title}\PY{p}{(} \PY{l+s+s2}{\PYZdq{}}\PY{l+s+s2}{Regression Results}\PY{l+s+s2}{\PYZdq{}}\PY{p}{)}
        \PY{n}{plt}\PY{o}{.}\PY{n}{legend}\PY{p}{(} \PY{p}{[}\PY{l+s+s1}{\PYZsq{}}\PY{l+s+s1}{Observed}\PY{l+s+s1}{\PYZsq{}}\PY{p}{,} \PY{l+s+s1}{\PYZsq{}}\PY{l+s+s1}{Predicted}\PY{l+s+s1}{\PYZsq{}}\PY{p}{]}\PY{p}{,} \PY{n}{loc}\PY{o}{=}\PY{l+s+s1}{\PYZsq{}}\PY{l+s+s1}{lower right}\PY{l+s+s1}{\PYZsq{}}\PY{p}{)}
        \PY{n}{plt}\PY{o}{.}\PY{n}{xlabel}\PY{p}{(} \PY{l+s+s2}{\PYZdq{}}\PY{l+s+s2}{Mean Air Temp (C)}\PY{l+s+s2}{\PYZdq{}}\PY{p}{)}
        \PY{n}{plt}\PY{o}{.}\PY{n}{ylabel}\PY{p}{(} \PY{l+s+s2}{\PYZdq{}}\PY{l+s+s2}{Max Stream Temperature (C)}\PY{l+s+s2}{\PYZdq{}}\PY{p}{)}
        \PY{n}{plt}\PY{o}{.}\PY{n}{text}\PY{p}{(} \PY{o}{\PYZhy{}}\PY{l+m+mi}{10}\PY{p}{,} \PY{l+m+mi}{23}\PY{p}{,} \PY{l+s+s2}{\PYZdq{}}\PY{l+s+s2}{r2=}\PY{l+s+s2}{\PYZdq{}} \PY{o}{+} \PY{l+s+s2}{\PYZdq{}}\PY{l+s+si}{\PYZob{}:.3\PYZcb{}}\PY{l+s+s2}{\PYZdq{}}\PY{o}{.}\PY{n}{format}\PY{p}{(} \PY{n}{r\PYZus{}value}\PY{o}{*}\PY{o}{*}\PY{l+m+mi}{2} \PY{p}{)}\PY{p}{,} \PY{n}{fontsize}\PY{o}{=}\PY{l+m+mi}{14} \PY{p}{)}
        \PY{n}{plt}\PY{o}{.}\PY{n}{show}\PY{p}{(}\PY{p}{)}
        \PY{c+c1}{\PYZsh{} show the plot}
        \PY{n}{plt}\PY{o}{.}\PY{n}{show}\PY{p}{(}\PY{p}{)}
\end{Verbatim}


    \begin{center}
    \adjustimage{max size={0.9\linewidth}{0.9\paperheight}}{output_9_0.png}
    \end{center}
    { \hspace*{\fill} \\}
    
    \subsubsection{Discussion of Results}\label{discussion-of-results}

Put your discussion for Problem 1 here.

    \subsection{Problem 2.}\label{problem-2.}

In this problem, we will essentially repeat Problem 1, but this time add
an additional explantory variable, "DayOfYear" to the regression. This
mean we will need to use multiple linear regression, rather than single
linear regression. Again, you are encouraged to review the "Linear
Regression in Python" PowerPoint on the course website. You will want to
use the \texttt{statsmodels} package. For outputs, be sure to include 1)
the best fit model parameters, and 2) the r-square value. Note that you
can copy/paste code from above if helpful!

    \subsubsection{Version 1 - Infer model from data
str}\label{version-1---infer-model-from-data-str}

    \begin{Verbatim}[commandchars=\\\{\}]
{\color{incolor}In [{\color{incolor}8}]:} \PY{c+c1}{\PYZsh{} additional imports not covered above}
        \PY{k+kn}{import} \PY{n+nn}{statsmodels}\PY{n+nn}{.}\PY{n+nn}{api} \PY{k}{as} \PY{n+nn}{sm}
        \PY{k+kn}{import} \PY{n+nn}{matplotlib}\PY{n+nn}{.}\PY{n+nn}{pyplot} \PY{k}{as} \PY{n+nn}{plt} 
        \PY{k+kn}{from} \PY{n+nn}{mpl\PYZus{}toolkits}\PY{n+nn}{.}\PY{n+nn}{mplot3d} \PY{k}{import} \PY{n}{Axes3D}
        
        \PY{c+c1}{\PYZsh{} using the previously loaded dataset, extract two columns of x data (one for each explanatory variable) and one for y}
        \PY{n}{XObs} \PY{o}{=} \PY{n}{df}\PY{p}{[} \PY{p}{[}\PY{l+s+s1}{\PYZsq{}}\PY{l+s+s1}{DayOfYear}\PY{l+s+s1}{\PYZsq{}}\PY{p}{,} \PY{l+s+s1}{\PYZsq{}}\PY{l+s+s1}{MaxAirTempC}\PY{l+s+s1}{\PYZsq{}} \PY{p}{]} \PY{p}{]} \PY{c+c1}{\PYZsh{} make a matrix, since we have two explanatory variables}
        \PY{n}{yObs} \PY{o}{=} \PY{n}{df}\PY{p}{[} \PY{l+s+s1}{\PYZsq{}}\PY{l+s+s1}{MaxStreamTempC}\PY{l+s+s1}{\PYZsq{}} \PY{p}{]}   \PY{c+c1}{\PYZsh{} response variable}
        
        \PY{c+c1}{\PYZsh{} first approach infers model from structure of data}
        \PY{n}{XObs} \PY{o}{=} \PY{n}{sm}\PY{o}{.}\PY{n}{add\PYZus{}constant}\PY{p}{(} \PY{n}{XObs} \PY{p}{)}    \PY{c+c1}{\PYZsh{} include an intercept term}
        \PY{n}{model} \PY{o}{=} \PY{n}{sm}\PY{o}{.}\PY{n}{OLS}\PY{p}{(} \PY{n}{yObs}\PY{p}{,} \PY{n}{XObs} \PY{p}{)}
        \PY{n}{est} \PY{o}{=} \PY{n}{model}\PY{o}{.}\PY{n}{fit}\PY{p}{(}\PY{p}{)}
        \PY{n+nb}{print}\PY{p}{(} \PY{n}{est}\PY{o}{.}\PY{n}{summary}\PY{p}{(}\PY{p}{)} \PY{p}{)}
        
        \PY{n+nb}{print}\PY{p}{(} \PY{l+s+s2}{\PYZdq{}}\PY{l+s+s2}{R2=}\PY{l+s+si}{\PYZob{}:.3\PYZcb{}}\PY{l+s+s2}{\PYZdq{}}\PY{o}{.}\PY{n}{format}\PY{p}{(} \PY{n}{est}\PY{o}{.}\PY{n}{rsquared} \PY{p}{)} \PY{p}{)}
        
        \PY{c+c1}{\PYZsh{} observed/predicted plot}
        \PY{n}{yModeled} \PY{o}{=} \PY{n}{est}\PY{o}{.}\PY{n}{params}\PY{p}{[}\PY{l+m+mi}{0}\PY{p}{]} \PY{o}{+} \PY{n}{est}\PY{o}{.}\PY{n}{params}\PY{p}{[}\PY{l+m+mi}{1}\PY{p}{]} \PY{o}{*} \PY{n}{XObs}\PY{o}{.}\PY{n}{DayOfYear}\PY{o}{.}\PY{n}{values} \PY{o}{+} \PY{n}{est}\PY{o}{.}\PY{n}{params}\PY{p}{[}\PY{l+m+mi}{2}\PY{p}{]} \PY{o}{*} \PY{n}{XObs}\PY{o}{.}\PY{n}{MaxAirTempC}\PY{o}{.}\PY{n}{values}
        
        \PY{n}{plt}\PY{o}{.}\PY{n}{plot}\PY{p}{(} \PY{n}{yModeled}\PY{p}{,} \PY{n}{yObs}\PY{p}{,} \PY{l+s+s1}{\PYZsq{}}\PY{l+s+s1}{o}\PY{l+s+s1}{\PYZsq{}}\PY{p}{,} \PY{n}{color}\PY{o}{=}\PY{l+s+s1}{\PYZsq{}}\PY{l+s+s1}{blue}\PY{l+s+s1}{\PYZsq{}} \PY{p}{)}    \PY{c+c1}{\PYZsh{} add modeled series}
        \PY{n}{plt}\PY{o}{.}\PY{n}{plot}\PY{p}{(} \PY{p}{[}\PY{l+m+mi}{0}\PY{p}{,}\PY{l+m+mi}{30}\PY{p}{]}\PY{p}{,} \PY{p}{[}\PY{l+m+mi}{0}\PY{p}{,}\PY{l+m+mi}{30}\PY{p}{]}\PY{p}{,} \PY{l+s+s1}{\PYZsq{}}\PY{l+s+s1}{\PYZhy{}}\PY{l+s+s1}{\PYZsq{}}\PY{p}{,} \PY{n}{color}\PY{o}{=}\PY{l+s+s1}{\PYZsq{}}\PY{l+s+s1}{red}\PY{l+s+s1}{\PYZsq{}} \PY{p}{)}     \PY{c+c1}{\PYZsh{} add line of perfect fit modeled series}
        \PY{n}{plt}\PY{o}{.}\PY{n}{title}\PY{p}{(} \PY{l+s+s2}{\PYZdq{}}\PY{l+s+s2}{Regression Results}\PY{l+s+s2}{\PYZdq{}}\PY{p}{)}
        \PY{n}{plt}\PY{o}{.}\PY{n}{xlabel}\PY{p}{(} \PY{l+s+s2}{\PYZdq{}}\PY{l+s+s2}{Predicted}\PY{l+s+s2}{\PYZdq{}}\PY{p}{)}
        \PY{n}{plt}\PY{o}{.}\PY{n}{ylabel}\PY{p}{(} \PY{l+s+s2}{\PYZdq{}}\PY{l+s+s2}{Observed}\PY{l+s+s2}{\PYZdq{}}\PY{p}{)}
        \PY{n}{plt}\PY{o}{.}\PY{n}{text}\PY{p}{(} \PY{l+m+mi}{5}\PY{p}{,} \PY{l+m+mi}{25}\PY{p}{,} \PY{l+s+s2}{\PYZdq{}}\PY{l+s+s2}{r2=}\PY{l+s+s2}{\PYZdq{}} \PY{o}{+} \PY{l+s+s2}{\PYZdq{}}\PY{l+s+si}{\PYZob{}:.3\PYZcb{}}\PY{l+s+s2}{\PYZdq{}}\PY{o}{.}\PY{n}{format}\PY{p}{(} \PY{n}{est}\PY{o}{.}\PY{n}{rsquared} \PY{p}{)}\PY{p}{,} \PY{n}{fontsize}\PY{o}{=}\PY{l+m+mi}{14} \PY{p}{)}
        \PY{c+c1}{\PYZsh{} show the plot}
        \PY{n}{plt}\PY{o}{.}\PY{n}{show}\PY{p}{(}\PY{p}{)}
\end{Verbatim}


    \begin{Verbatim}[commandchars=\\\{\}]
                            OLS Regression Results                            
==============================================================================
Dep. Variable:         MaxStreamTempC   R-squared:                       0.752
Model:                            OLS   Adj. R-squared:                  0.752
Method:                 Least Squares   F-statistic:                     9104.
Date:                Mon, 08 Oct 2018   Prob (F-statistic):               0.00
Time:                        15:30:35   Log-Likelihood:                -14413.
No. Observations:                6020   AIC:                         2.883e+04
Df Residuals:                    6017   BIC:                         2.885e+04
Df Model:                           2                                         
Covariance Type:            nonrobust                                         
===============================================================================
                  coef    std err          t      P>|t|      [0.025      0.975]
-------------------------------------------------------------------------------
const           1.4539      0.102     14.243      0.000       1.254       1.654
DayOfYear       0.0093      0.000     28.899      0.000       0.009       0.010
MaxAirTempC     0.5607      0.004    130.109      0.000       0.552       0.569
==============================================================================
Omnibus:                       30.922   Durbin-Watson:                   0.287
Prob(Omnibus):                  0.000   Jarque-Bera (JB):               32.147
Skew:                           0.155   Prob(JB):                     1.05e-07
Kurtosis:                       3.181   Cond. No.                         646.
==============================================================================

Warnings:
[1] Standard Errors assume that the covariance matrix of the errors is correctly specified.
R2=0.752

    \end{Verbatim}

    
    \begin{verbatim}
<matplotlib.figure.Figure at 0x7fcab0b59ac8>
    \end{verbatim}

    
    \subsubsection{Version 2 - Formula
appraoch}\label{version-2---formula-appraoch}

    \begin{Verbatim}[commandchars=\\\{\}]
{\color{incolor}In [{\color{incolor}7}]:} \PY{k+kn}{import} \PY{n+nn}{numpy} \PY{k}{as} \PY{n+nn}{np} 
        \PY{k+kn}{import} \PY{n+nn}{statsmodels}\PY{n+nn}{.}\PY{n+nn}{formula}\PY{n+nn}{.}\PY{n+nn}{api} \PY{k}{as} \PY{n+nn}{smf}
        
        \PY{c+c1}{\PYZsh{} formula\PYZhy{}based approach}
        \PY{n}{model} \PY{o}{=} \PY{n}{smf}\PY{o}{.}\PY{n}{ols}\PY{p}{(} \PY{n}{formula}\PY{o}{=}\PY{l+s+s1}{\PYZsq{}}\PY{l+s+s1}{MaxStreamTempC \PYZti{} DayOfYear + MaxAirTempC}\PY{l+s+s1}{\PYZsq{}}\PY{p}{,} \PY{n}{data}\PY{o}{=}\PY{n}{df}\PY{p}{)}
        \PY{n}{est} \PY{o}{=} \PY{n}{model}\PY{o}{.}\PY{n}{fit}\PY{p}{(}\PY{p}{)}
        \PY{n+nb}{print}\PY{p}{(} \PY{n}{est}\PY{o}{.}\PY{n}{summary}\PY{p}{(}\PY{p}{)} \PY{p}{)}
        
        \PY{c+c1}{\PYZsh{} This time lets create the 3d plot showing results}
        \PY{c+c1}{\PYZsh{} make DayOfYear, MaxAirTempC grid for 3D plot}
        \PY{n}{xx1}\PY{p}{,} \PY{n}{xx2} \PY{o}{=} \PY{n}{np}\PY{o}{.}\PY{n}{meshgrid}\PY{p}{(}\PY{n}{np}\PY{o}{.}\PY{n}{linspace}\PY{p}{(}\PY{n}{XObs}\PY{o}{.}\PY{n}{DayOfYear}\PY{o}{.}\PY{n}{min}\PY{p}{(}\PY{p}{)}\PY{p}{,} \PY{n}{XObs}\PY{o}{.}\PY{n}{DayOfYear}\PY{o}{.}\PY{n}{max}\PY{p}{(}\PY{p}{)}\PY{p}{,} \PY{l+m+mi}{100}\PY{p}{)}\PY{p}{,}
                               \PY{n}{np}\PY{o}{.}\PY{n}{linspace}\PY{p}{(}\PY{n}{XObs}\PY{o}{.}\PY{n}{MaxAirTempC}\PY{o}{.}\PY{n}{min}\PY{p}{(}\PY{p}{)}\PY{p}{,} \PY{n}{XObs}\PY{o}{.}\PY{n}{MaxAirTempC}\PY{o}{.}\PY{n}{max}\PY{p}{(}\PY{p}{)}\PY{p}{,} \PY{l+m+mi}{100}\PY{p}{)}\PY{p}{)}
        \PY{c+c1}{\PYZsh{} plot the hyperplane by evaluating the parameters on the grid}
        \PY{n}{Z} \PY{o}{=} \PY{n}{est}\PY{o}{.}\PY{n}{params}\PY{p}{[}\PY{l+m+mi}{0}\PY{p}{]} \PY{o}{+} \PY{n}{est}\PY{o}{.}\PY{n}{params}\PY{p}{[}\PY{l+m+mi}{1}\PY{p}{]} \PY{o}{*} \PY{n}{xx1} \PY{o}{+} \PY{n}{est}\PY{o}{.}\PY{n}{params}\PY{p}{[}\PY{l+m+mi}{2}\PY{p}{]} \PY{o}{*} \PY{n}{xx2}
        
        \PY{c+c1}{\PYZsh{} create matplotlib 3d axes}
        \PY{n}{fig} \PY{o}{=} \PY{n}{plt}\PY{o}{.}\PY{n}{figure}\PY{p}{(}\PY{n}{figsize}\PY{o}{=}\PY{p}{(}\PY{l+m+mi}{12}\PY{p}{,} \PY{l+m+mi}{8}\PY{p}{)}\PY{p}{)}
        \PY{n}{ax} \PY{o}{=} \PY{n}{fig}\PY{o}{.}\PY{n}{add\PYZus{}subplot}\PY{p}{(}\PY{l+m+mi}{111}\PY{p}{,}\PY{n}{projection}\PY{o}{=}\PY{l+s+s1}{\PYZsq{}}\PY{l+s+s1}{3d}\PY{l+s+s1}{\PYZsq{}}\PY{p}{)}  \PY{c+c1}{\PYZsh{} Axes3D(fig, azim=\PYZhy{}115, elev=15)}
        
        \PY{c+c1}{\PYZsh{} plot hyperplane}
        \PY{n}{surf} \PY{o}{=} \PY{n}{ax}\PY{o}{.}\PY{n}{plot\PYZus{}surface}\PY{p}{(}\PY{n}{xx1}\PY{p}{,} \PY{n}{xx2}\PY{p}{,} \PY{n}{Z}\PY{p}{,} \PY{n}{cmap}\PY{o}{=}\PY{n}{plt}\PY{o}{.}\PY{n}{cm}\PY{o}{.}\PY{n}{RdBu\PYZus{}r}\PY{p}{,} \PY{n}{alpha}\PY{o}{=}\PY{l+m+mf}{0.6}\PY{p}{,} \PY{n}{linewidth}\PY{o}{=}\PY{l+m+mi}{0}\PY{p}{)}
        
        \PY{c+c1}{\PYZsh{} plot data points \PYZhy{} points over the HP are white, points below are black}
        \PY{n}{resid} \PY{o}{=} \PY{n}{yObs} \PY{o}{\PYZhy{}} \PY{n}{est}\PY{o}{.}\PY{n}{predict}\PY{p}{(}\PY{n}{XObs}\PY{p}{)}
        \PY{n}{ax}\PY{o}{.}\PY{n}{scatter}\PY{p}{(}\PY{n}{XObs}\PY{p}{[}\PY{n}{resid} \PY{o}{\PYZgt{}}\PY{o}{=} \PY{l+m+mi}{0}\PY{p}{]}\PY{o}{.}\PY{n}{DayOfYear}\PY{p}{,} \PY{n}{XObs}\PY{p}{[}\PY{n}{resid} \PY{o}{\PYZgt{}}\PY{o}{=} \PY{l+m+mi}{0}\PY{p}{]}\PY{o}{.}\PY{n}{MaxAirTempC}\PY{p}{,} \PY{n}{yObs}\PY{p}{[}\PY{n}{resid} \PY{o}{\PYZgt{}}\PY{o}{=} \PY{l+m+mi}{0}\PY{p}{]}\PY{p}{,} \PY{n}{color}\PY{o}{=}\PY{l+s+s1}{\PYZsq{}}\PY{l+s+s1}{r}\PY{l+s+s1}{\PYZsq{}}\PY{p}{,} \PY{n}{alpha}\PY{o}{=}\PY{l+m+mf}{1.0}\PY{p}{,} \PY{n}{facecolor}\PY{o}{=}\PY{l+s+s1}{\PYZsq{}}\PY{l+s+s1}{white}\PY{l+s+s1}{\PYZsq{}}\PY{p}{)}
        \PY{n}{ax}\PY{o}{.}\PY{n}{scatter}\PY{p}{(}\PY{n}{XObs}\PY{p}{[}\PY{n}{resid} \PY{o}{\PYZlt{}} \PY{l+m+mi}{0}\PY{p}{]}\PY{o}{.}\PY{n}{DayOfYear}\PY{p}{,} \PY{n}{XObs}\PY{p}{[}\PY{n}{resid} \PY{o}{\PYZlt{}} \PY{l+m+mi}{0}\PY{p}{]}\PY{o}{.}\PY{n}{MaxAirTempC}\PY{p}{,} \PY{n}{yObs}\PY{p}{[}\PY{n}{resid} \PY{o}{\PYZlt{}} \PY{l+m+mi}{0}\PY{p}{]}\PY{p}{,} \PY{n}{color}\PY{o}{=}\PY{l+s+s1}{\PYZsq{}}\PY{l+s+s1}{b}\PY{l+s+s1}{\PYZsq{}}\PY{p}{,} \PY{n}{alpha}\PY{o}{=}\PY{l+m+mf}{1.0}\PY{p}{)}
        
        \PY{c+c1}{\PYZsh{} set axis labels}
        \PY{n}{ax}\PY{o}{.}\PY{n}{set\PYZus{}xlabel}\PY{p}{(}\PY{l+s+s1}{\PYZsq{}}\PY{l+s+s1}{Day Of Year}\PY{l+s+s1}{\PYZsq{}}\PY{p}{)}
        \PY{n}{ax}\PY{o}{.}\PY{n}{set\PYZus{}ylabel}\PY{p}{(}\PY{l+s+s1}{\PYZsq{}}\PY{l+s+s1}{Mean Air Temp (C)}\PY{l+s+s1}{\PYZsq{}}\PY{p}{)}
        \PY{n}{ax}\PY{o}{.}\PY{n}{set\PYZus{}zlabel}\PY{p}{(}\PY{l+s+s1}{\PYZsq{}}\PY{l+s+s1}{Mean Stream Temp (C)}\PY{l+s+s1}{\PYZsq{}}\PY{p}{)}
        
        \PY{n}{plt}\PY{o}{.}\PY{n}{show}\PY{p}{(}\PY{p}{)}
\end{Verbatim}


    \begin{Verbatim}[commandchars=\\\{\}]
                            OLS Regression Results                            
==============================================================================
Dep. Variable:         MaxStreamTempC   R-squared:                       0.752
Model:                            OLS   Adj. R-squared:                  0.752
Method:                 Least Squares   F-statistic:                     9104.
Date:                Sun, 08 Oct 2017   Prob (F-statistic):               0.00
Time:                        10:19:53   Log-Likelihood:                -14413.
No. Observations:                6020   AIC:                         2.883e+04
Df Residuals:                    6017   BIC:                         2.885e+04
Df Model:                           2                                         
Covariance Type:            nonrobust                                         
===============================================================================
                  coef    std err          t      P>|t|      [0.025      0.975]
-------------------------------------------------------------------------------
Intercept       1.4539      0.102     14.243      0.000       1.254       1.654
DayOfYear       0.0093      0.000     28.899      0.000       0.009       0.010
MaxAirTempC     0.5607      0.004    130.109      0.000       0.552       0.569
==============================================================================
Omnibus:                       30.922   Durbin-Watson:                   0.287
Prob(Omnibus):                  0.000   Jarque-Bera (JB):               32.147
Skew:                           0.155   Prob(JB):                     1.05e-07
Kurtosis:                       3.181   Cond. No.                         646.
==============================================================================

Warnings:
[1] Standard Errors assume that the covariance matrix of the errors is correctly specified.

    \end{Verbatim}

    \begin{center}
    \adjustimage{max size={0.9\linewidth}{0.9\paperheight}}{output_15_1.png}
    \end{center}
    { \hspace*{\fill} \\}
    
    \subsection{Problem 3.}\label{problem-3.}

Using the Fish Presence/Absence Dataset, estimate parameters for a
Logistic Model. The model should be able to predict whether or not
Cutthroat Trout were absent or present based on the quantity of Large
Woody Debris (LWD in the dataset, kg/m2) in proximity to the sample
(indicated by 0 (absence) or 1 (presence) in the CUTT field in the
dataset. The model should be able to predict the likelihood of finding
cutthroat trout based on level of large woody debris at the site.

To get the data, use pandas. To run the regression, use the
LogisticRegression() function from the sklearn package.

    \begin{Verbatim}[commandchars=\\\{\}]
{\color{incolor}In [{\color{incolor}1}]:} \PY{k+kn}{from} \PY{n+nn}{pandas} \PY{k}{import} \PY{n}{read\PYZus{}excel}
        
        \PY{k+kn}{from} \PY{n+nn}{sklearn}\PY{n+nn}{.}\PY{n+nn}{linear\PYZus{}model} \PY{k}{import} \PY{n}{LogisticRegression} 
        \PY{k+kn}{import} \PY{n+nn}{numpy} \PY{k}{as} \PY{n+nn}{np}
        \PY{k+kn}{import} \PY{n+nn}{matplotlib}\PY{n+nn}{.}\PY{n+nn}{pyplot} \PY{k}{as} \PY{n+nn}{plt}
        
        \PY{n}{df} \PY{o}{=} \PY{n}{read\PYZus{}excel}\PY{p}{(} \PY{l+s+s1}{\PYZsq{}}\PY{l+s+s1}{http://explorer.bee.oregonstate.edu/Topic/Modeling/Data/FishPresenceAbsence.xlsx}\PY{l+s+s1}{\PYZsq{}} \PY{p}{)}
        
        \PY{n}{df} \PY{o}{=} \PY{n}{df}\PY{o}{.}\PY{n}{sort\PYZus{}values}\PY{p}{(}\PY{l+s+s1}{\PYZsq{}}\PY{l+s+s1}{LWD Density}\PY{l+s+s1}{\PYZsq{}}\PY{p}{)}    \PY{c+c1}{\PYZsh{} this is mostly for convenience when plotting}
        
        \PY{c+c1}{\PYZsh{} get one column of data \PYZhy{} the first}
        \PY{n}{X} \PY{o}{=} \PY{n}{df}\PY{p}{[}\PY{l+s+s1}{\PYZsq{}}\PY{l+s+s1}{LWD Density}\PY{l+s+s1}{\PYZsq{}}\PY{p}{]}\PY{o}{.}\PY{n}{values}        \PY{c+c1}{\PYZsh{} 1D array \PYZhy{} rows=observation, col = pH values}
        \PY{n}{y} \PY{o}{=} \PY{n}{df}\PY{p}{[}\PY{l+s+s1}{\PYZsq{}}\PY{l+s+s1}{CUTT}\PY{l+s+s1}{\PYZsq{}}\PY{p}{]}\PY{o}{.}\PY{n}{values}    \PY{c+c1}{\PYZsh{} an array of the correct classification for each observation}
        
        \PY{n}{Xm} \PY{o}{=} \PY{n}{X}\PY{o}{.}\PY{n}{reshape}\PY{p}{(} \PY{o}{\PYZhy{}}\PY{l+m+mi}{1}\PY{p}{,}\PY{l+m+mi}{1} \PY{p}{)}      \PY{c+c1}{\PYZsh{} convert array to 2D matrix, since this is required for LogisticRegression() call}
        
        \PY{c+c1}{\PYZsh{} make and fit a model}
        \PY{n}{logisticModel} \PY{o}{=} \PY{n}{LogisticRegression}\PY{p}{(}\PY{p}{)}
        \PY{n}{logisticModel}\PY{o}{.}\PY{n}{fit}\PY{p}{(}\PY{n}{Xm}\PY{p}{,}\PY{n}{y}\PY{p}{)}
        
        \PY{n}{predictions} \PY{o}{=} \PY{n}{logisticModel}\PY{o}{.}\PY{n}{predict\PYZus{}proba}\PY{p}{(}\PY{n}{Xm}\PY{p}{)}
        
        \PY{n}{plt}\PY{o}{.}\PY{n}{plot}\PY{p}{(}\PY{n}{X}\PY{p}{,}\PY{n}{y}\PY{p}{,}\PY{l+s+s1}{\PYZsq{}}\PY{l+s+s1}{o}\PY{l+s+s1}{\PYZsq{}}\PY{p}{)}
        \PY{n}{plt}\PY{o}{.}\PY{n}{plot}\PY{p}{(}\PY{n}{X}\PY{p}{,}\PY{n}{predictions}\PY{p}{[}\PY{p}{:}\PY{p}{,}\PY{l+m+mi}{1}\PY{p}{]}\PY{p}{,}\PY{l+s+s1}{\PYZsq{}}\PY{l+s+s1}{\PYZhy{}}\PY{l+s+s1}{\PYZsq{}}\PY{p}{)}     \PY{c+c1}{\PYZsh{} note that \PYZdq{}absence\PYZdq{} probabilities are in the first coluumn, \PYZdq{}Presence\PYZdq{} is in the second }
        \PY{n}{plt}\PY{o}{.}\PY{n}{xlabel}\PY{p}{(} \PY{l+s+s2}{\PYZdq{}}\PY{l+s+s2}{LWD}\PY{l+s+s2}{\PYZdq{}} \PY{p}{)}
        \PY{n}{plt}\PY{o}{.}\PY{n}{ylabel}\PY{p}{(} \PY{l+s+s2}{\PYZdq{}}\PY{l+s+s2}{Fish Presence Probability}\PY{l+s+s2}{\PYZdq{}}\PY{p}{)}
        \PY{n}{plt}\PY{o}{.}\PY{n}{show}\PY{p}{(}\PY{p}{)}
        
        \PY{c+c1}{\PYZsh{} output model coefficients \PYZhy{} because there is one input, we will have one model coefficent (+ intercept)}
        \PY{n+nb}{print}\PY{p}{(} \PY{n}{logisticModel}\PY{o}{.}\PY{n}{coef\PYZus{}} \PY{p}{)}
        \PY{n+nb}{print}\PY{p}{(} \PY{n}{logisticModel}\PY{o}{.}\PY{n}{intercept\PYZus{}} \PY{p}{)}
        \PY{n}{slope} \PY{o}{=} \PY{n}{logisticModel}\PY{o}{.}\PY{n}{coef\PYZus{}}\PY{p}{[}\PY{l+m+mi}{0}\PY{p}{]}\PY{p}{[}\PY{l+m+mi}{0}\PY{p}{]}
        \PY{n}{intercept} \PY{o}{=} \PY{n}{logisticModel}\PY{o}{.}\PY{n}{intercept\PYZus{}}\PY{p}{[}\PY{l+m+mi}{0}\PY{p}{]}
        \PY{n+nb}{print}\PY{p}{(} \PY{l+s+s2}{\PYZdq{}}\PY{l+s+s2}{Slope: }\PY{l+s+si}{\PYZob{}:.3\PYZcb{}}\PY{l+s+s2}{,  Intercept: }\PY{l+s+si}{\PYZob{}:.3\PYZcb{}}\PY{l+s+s2}{\PYZdq{}}\PY{o}{.}\PY{n}{format}\PY{p}{(}\PY{n}{slope}\PY{p}{,} \PY{n}{intercept}\PY{p}{)}\PY{p}{)}
        
        \PY{n}{count} \PY{o}{=} \PY{l+m+mi}{0}
        \PY{k}{for} \PY{n}{i} \PY{o+ow}{in} \PY{n+nb}{range}\PY{p}{(}\PY{l+m+mi}{0}\PY{p}{,}\PY{n+nb}{len}\PY{p}{(}\PY{n}{y}\PY{p}{)}\PY{p}{)}\PY{p}{:}
            
            \PY{n}{pred} \PY{o}{=} \PY{n}{logisticModel}\PY{o}{.}\PY{n}{predict}\PY{p}{(} \PY{n}{X}\PY{p}{[}\PY{n}{i}\PY{p}{]}\PY{p}{)}
            \PY{n}{yObs} \PY{o}{=} \PY{n}{y}\PY{p}{[}\PY{n}{i}\PY{p}{]}
            
            \PY{k}{if} \PY{n}{pred} \PY{o}{==} \PY{n}{yObs}\PY{p}{:}
                \PY{n}{count} \PY{o}{=} \PY{n}{count} \PY{o}{+} \PY{l+m+mi}{1}
                
        \PY{n+nb}{print}\PY{p}{(} \PY{l+s+s2}{\PYZdq{}}\PY{l+s+s2}{performance=}\PY{l+s+s2}{\PYZdq{}}\PY{p}{,} \PY{n}{count}\PY{o}{/}\PY{n+nb}{len}\PY{p}{(}\PY{n}{y}\PY{p}{)} \PY{p}{)}
\end{Verbatim}


    \begin{center}
    \adjustimage{max size={0.9\linewidth}{0.9\paperheight}}{output_17_0.png}
    \end{center}
    { \hspace*{\fill} \\}
    
    \begin{Verbatim}[commandchars=\\\{\}]
[[ 0.3429609]]
[-1.91277305]
Slope: 0.343,  Intercept: -1.91
performance= 0.8

    \end{Verbatim}

    \subsubsection{When you are done, save this notebook to your hub
directory, "download" it to your local machine, and upload it to Canvas.
When saving the notebook, please name it HW2-LinearRegression-\{your
last
name\}.ipynb}\label{when-you-are-done-save-this-notebook-to-your-hub-directory-download-it-to-your-local-machine-and-upload-it-to-canvas.-when-saving-the-notebook-please-name-it-hw2-linearregression-your-last-name.ipynb}


    % Add a bibliography block to the postdoc
    
    
    
    \end{document}
